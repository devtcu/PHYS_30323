\documentclass{article}
\usepackage{graphicx} % Required for inserting images
\usepackage{amsmath}
\usepackage{xcolor}
\usepackage{amssymb}




\begin{document}

{\large PHYS 20323/60323: Fall 2023 - LaTeX Example}

\vskip0.1in
\noindent {\textbf{1. The following questions refer to stars in the Table below.}} \\
Note: There may be multiple answers

\vskip0.1in

\begin{center}
\begin{tabular}{|c|c|c|c|c|c|}\hline
Name & Mass & Luminosity & Lifetime & Temperature & Radius\\\hline\hline
$\eta$ Car.  & 60. M_\odot &  10^6 L_\odot  & 8.5 \times 10^5 years &  &   \\\hline 
$\epsilon$ Eri.   & 6.0 M_\odot  &  10^3 L_\odot &  & 20,000 K &   \\\hline
$\delta$ Scu.   & 2.0 M_\odot   &   & 5.0 \times 10^8 years &  &  2 R_\odot\\\hline
$\beta$ Cyg.   & 1.3 M_\odot   &  3.5 L_\odot &  &  &   \\\hline
$\alpha$ Cen.   & 1.0 M_\odot   &   &  &  &  1 R_\odot \\\hline
$\gamma$ Del.   & 0.7 M_\odot   &   & 4.5 \times 10^{10} years & 5000 K &   \\\hline
\end{tabular}\vskip 0.2in
\end{center}

\vskip0.1in

(a) (4 points) Which of these stars will produce a planetary nebula.
\vskip0.1in
(b) (4 points) Elements heavier than \textit{Carbon} will be produced in which stars.

\vskip0.1in

\noindent {2. An electron is found to be in the spin state (in the \textit{z}-basis):} \textit{X} = A
$
\begin{pmatrix} 
	3i \\
	4 \\
	\end{pmatrix}
$
\vskip0.1in
(a) (5 points) Determine the possible values of A such that the state is normalized.
\vskip0.1in
(b) (5 points) Find the expectation values of the operators {$\color{red}S_x, \;\color{purple}S_y, \;\color{orange}S_z \color{black}\; and \; \Vec{S^2}.$}
\vskip0.1in
The matrix representations in the \textit{z}-basis for the components of electron spin operators are
given by:
\vskip0.1in

$$
    \color{red}
    S_x = \frac{\hbar}{2}
    \begin{pmatrix} 
    	0 & 1 \\
    	1 & 0 \\
    	\end{pmatrix}
     ; \qquad
     \color{purple}
     S_y = \frac{\hbar}{2}
     \begin{pmatrix} 
    	0 & -i \\
    	i & 0 \\
    	\end{pmatrix}
     ; \qquad
     \color{orange}
     S_z = \frac{\hbar}{2}
     \begin{pmatrix} 
    	1 & 0 \\
    	0 & -1 \\
    	\end{pmatrix}
$$

\noindent {3. The average electrostatic field in the earth’s atmosphere in fair weather is approximately given:} \\
$$
\Vec{E}=E_0(Ae^{\alpha z} + Be^{\beta z})\;\hat{z},
$$

\noindent {where A, B, $\alpha$, $\beta$ are positive constants and \textit{z} is the height above the (locally flat) earth surface.} \\
\vskip0.1in

(a) (5 points) Find the average charge density in the atmosphere as a function of height
\vskip0.1in

(b) (5 points) Find the electric potential as a function height above the earth.


\end{document}
